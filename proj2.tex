\documentclass[twocolumn, 11pt, a4paper]{article}
\usepackage[left=14mm, top=23mm, text={183mm, 252mm}]{geometry}
\usepackage[utf8]{inputenc}
\usepackage[T1]{fontenc}
\usepackage{lmodern}
\usepackage[czech]{babel}
\usepackage{amsthm, amssymb, amsmath}

\theoremstyle{definition}
\newtheorem{definition}{Definice}

\theoremstyle{plain}
\newtheorem{algorithm}[definition]{Algoritmus}

\begin{document}
    \begin{titlepage}
        \begin{center}
            \textsc{\Huge{Vysoké učení technické v Brně}\\[0.5em]
            \huge{Fakulta informačních technologií}}\\
            \vspace{\stretch{0.382}}
            \LARGE
            Typografie a publikování -- 2. projekt\\[0.5em]
            Sazba dokumentů a matematických výrazů\\
            \vspace{\stretch{0.618}}
        \end{center}
        {\LARGE 2024 \hfill
        Valentyn Vorobec (xvorob02)}
    \end{titlepage}

    \section*{Úvod}
    V této úloze si vysázíme titulní stranu a kousek matematického textu, v němž se vyskytují například Definice \ref{kon_pre_str} nebo rovnice (\ref{rce_2}) na straně \pageref{rce_2}. Pro vytvoření těchto odkazů používáme kombinace příkazů \verb|\label|, \verb|\ref|, \verb|\eqref| a \verb|\pageref|. Před odkazy patří nezlomitelná mezera. Pro zvýrazňování textu se používají příkazy \verb|\verb| a \verb|\emph|.
    
    Titulní strana je vysázena prostředím \texttt{titlepage} a~nadpis je v optickém středu s využitím \emph{přesného} zlatého řezu, který byl probrán na přednášce. Dále jsou na titulní straně čtyři různé velikosti písma a mezi dvojicemi řádků textu je použito řádkování se zadanou relativní velikostí 0,5\,em a 0,6\,em\footnote{Použijte správný typ mezery mezi číslem a jednotkou.}.
    \section{Matematický text}
    Matematické symboly a výrazy v plynulém textu jsou v prostředí \texttt{math.} Definice a věty sázíme v prostředí definovaném příkazem \verb|\newtheorem| z balíku \texttt{amsthm.} Tato prostředí obracejí význam \verb|\emph:| uvnitř textu sázeného kurzívou se zvýrazňuje písmem v základním řezu. Někdy je vhodné použít konstrukci \verb|${}$| nebo \verb|\mbox{}|, která zabrání zalomení (matematického) textu. Pozor také na tvar i sklon řeckých písmen: srovnejte \verb|\epsilon| a \verb|\varepsilon|, \verb|\Xi| a \verb|\varXi.|
    
    \begin{definition}
        \label{kon_pre_str}
        Konečný přepisovací stroj \emph{neboli} Mealyho automat \emph{je definován jako uspořádaná pětice tvaru $M = (Q, \varSigma, \varGamma, \delta, q_0)$, kde:}
        \begin{itemize}
            \item \emph{Q je konečná množina} stavů,
            \item $\varSigma$ \emph{je konečná vstupní} abeceda,
            \item $\varGamma$ \emph{je konečná výstupní} abeceda,
            \item $\delta : Q \times \varSigma \rightarrow Q \times \varGamma$ \emph{je totální} přechodová funkce,
            \item $q_0 \in Q$ \emph{je} počáteční stav.
        \end{itemize}
    \end{definition}
    
    \subsection{Podsekce s definicí}
    Pomocí přechodové funkce $\delta$ zavedeme novou funkci~$\delta^\ast$ pro překlad vstupních slov $u \in \varSigma^\ast$ do výstupních slov $w \in \varGamma^\ast$.
    \begin{definition}
        \label{def_mealyho_automat}
         \emph{Nechť $M = (Q, \varSigma, \varGamma, \delta, q_0)$ je Mealyho automat.} Překládací funkce ${\delta^\ast : Q \times \varSigma^\ast \times \varGamma^\ast \rightarrow \varGamma^\ast}$ \emph{je~pro každý stav ${q \in Q}$, symbol $x \in \varSigma$, slova ${u \in \varSigma^\ast}$, ${w \in \varGamma^\ast}$ definována rekurentním předpisem:}

         \begin{itemize}
            \item $\delta^\ast(q, \varepsilon, w) = w$
            \item $\delta^\ast(q, xu, w) = \delta^\ast(q^\prime, u, wy), kde (q^\prime, y) = \delta(q, x)$
        \end{itemize}
    \end{definition}

    \subsection{Rovnice}
    Složitější matematické formule sázíme mimo plynulý text pomocí prostředí \texttt{displaymath.} Lze umístit i více výrazů na jeden řádek, ale pak je třeba tyto vhodně oddělit, například pomocí \verb|\quad|, při dostatku místa i~\verb|\qquad|.
    $$
        g^a_n \notin A^{B^n}
        \qquad
        y^1_0 - \sqrt[5]{x + \sqrt[7]{y}}
        \qquad
        x > y^2 \geq y^3
    $$

    Velikost závorek a svislých čar je potřeba přizpůsobit jejich obsahu. Velikost lze stanovit explicitně, anebo pomocí \verb|\left| a \verb|\right.| Kombinační čísla sázejte makrem \verb|\binom.|
    $$
        \left|\bigcup P\right| = \sum\limits _{\emptyset \neq X \subseteq P} (-1)^{|X|-1}\left|\bigcap X\right|
    $$

    $$
        F_{n+1} = \binom{n}{0} + \binom{n - 1}{1} + \binom{n - 2}{2} + \dots + \binom{\lceil \frac{n}{2} \rceil}{\lfloor \frac{n}{2} \rfloor}
    $$

    V rovnici (\ref{rce_1}) jsou tři typy závorek s různou \emph{explicitně} definovanou velikostí. Obě rovnice mají svisle zarovnaná rovnítka. Použijte k tomu vhodné prostředí.
    \begin{eqnarray}
        \label{rce_1}
        \biggl(\Bigl\{ b \otimes [c_1 \oplus c_2] \circ a\Bigr\}^{\frac{2}{3}}\biggl) & = & \log_zx \\
        \label{rce_2}
        \int_a^b f(x)\,\mathrm{d}x & = & -\int_b^a f(y)\,\mathrm{d}y
    \end{eqnarray}
    \noindent
    V této větě vidíme, jak se vysází proměnná určující limitu v běžném textu: $\lim_{m\rightarrow \infty}f(m)$. Podobně je to i s dalšími symboly jako $\bigcup_{N \in \mathcal{M}}N$ či $\sum _{i=1}^m x^2_i$. S vynucením méně úsporné sazby příkazem \verb|\limits| budou vzorce vysázeny v podobě $\lim\limits_{m\rightarrow \infty}f(m)$ a $\sum\limits _{i=1}^m x^2_i$.
    
    \section{Matice}
    Pro sázení matic se používá prostředí \texttt{array} a závorky s výškou nastavenou pomocí \verb|\left|, \verb|\right|.
    $$
        D = \left|
        \begin{array}{cccc}
            a_{11} & a_{12} & \ldots & a_{1n} \\
            a_{21} & a_{22} & \ldots & a_{2n} \\
            \vdots & \vdots & \ddots & \vdots \\
            a_{m1} & a_{m2} & \ldots & a_{mn}
        \end{array}
        \right|
        =
        \left|
        \begin{array}{cc}
            x & y \\
            t & w
        \end{array}
        \right|
        = xw - yt
    $$

    Prostředí \texttt{array} lze úspěšně využít i jinde, například na pravé straně následující rovnosti.
    $$
        \binom{n}{k} = \left\{\begin{array}{ll}
             \frac{n!}{k!(n-k)!} & \text{pro } 0 \leq k \leq n \\
             0 & \text{jinak}
        \end{array} \right.
    $$
\end{document}
